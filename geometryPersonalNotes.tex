\documentclass[12pt]{book}

\usepackage[utf8]{inputenc}
\usepackage{amsmath, amssymb, amsfonts, amsbsy, amsthm, latexsym, stmaryrd, mathrsfs}
\usepackage{listings} % Used to display code.
\usepackage{enumerate} % Make lists.
\usepackage{hyperref} % Hyperlinks 
\usepackage{color, xcolor} % Colors!
\usepackage{fullpage} % Change page setup.
\usepackage{setspace}
\usepackage{float} % Used for moving images
\usepackage{wrapfig} % Used for floating images
\usepackage{relsize} % Make mathprint larger
\usepackage{graphicx} % Include graphics.
\usepackage{booktabs} % Tables
\usepackage{braket} % Brakets
\usepackage{fourier} % math & rm
\usepackage[scaled=0.875]{helvet} % ss
\usepackage{cancel} % Strikethrough
\renewcommand{\ttdefault}{lmtt} %tt

%\setlength\parindent{0pt} %This command sets the paragraph indentation to 0.

% Polynomial Long division
\usepackage{mathtools}

% Colors (use with the package ``color'')

\definecolor{orange}{rgb}{1,0.5,0}
\definecolor{purple}{rgb}{.5,0,0.5}
\definecolor{dgreen}{rgb}{.1, .7, .2}
\definecolor{codegreen}{rgb}{0,0.6,0}
\definecolor{codegray}{rgb}{0.5,0.5,0.5}
\definecolor{codepurple}{rgb}{0.58,0,0.82}
\definecolor{backcolour}{rgb}{0.95,0.95,0.92}
 
\lstdefinestyle{mystyle}{
    backgroundcolor=\color{backcolour},   
    commentstyle=\color{codegreen},
    keywordstyle=\color{magenta},
    numberstyle=\tiny\color{codegray},
    stringstyle=\color{codepurple},
    basicstyle=\footnotesize,
    breakatwhitespace=false,         
    breaklines=true,                 
    captionpos=b,                    
    keepspaces=true,                 
    numbers=left,                    
    numbersep=5pt,                  
    showspaces=false,                
    showstringspaces=false,
    showtabs=false,                  
    tabsize=2
}

\lstset{style=mystyle}

% Short-cut symbols

\def\C{{\mathbb{C}}}
\def\N{{\mathbb{N}}}
\def\Q{{\mathbb{Q}}}
\def\R{{\mathbb{R}}}
\def\Z{{\mathbb{Z}}}
\def\l{{\ell}}
\def\U{{\mathbb{U}}}
\def\varep{{\varepsilon}}
\def\Zn[#1]{{\Z/#1\Z}}
\def\Un[#1]{{U(\Zn[#1])}}
\def\eq[#1]{{\overline{#1}}}
\def\lcm{{\text{lcm}}}
\newcommand{\specialcell}[2][c]{%
  \begin{tabular}[#1]{@{}c@{}}#2\end{tabular}}

% Fancy symbols
\def\sx{{\mathsf{x}}}
\def\sy{{\mathsf{y}}}
\def\sz{{\mathsf{z}}}
\def\sf{{\mathsf{f}}}
\def\sg{{\mathsf{g}}}
\def\sa{{\mathsf{a}}}
\def\sb{{\mathsf{b}}}
\def\sc{{\mathsf{c}}}
\def\sd{{\mathsf{d}}}
\def\se{{\mathsf{e}}}
\def\si{{\mathsf{i}}}
\def\sj{{\mathsf{j}}}
\def\sk{{\mathsf{k}}}
\def\sl{{\mathsf{l}}}
\def\sm{{\mathsf{m}}}
\def\sn{{\mathsf{n}}}

\def\su{{\mathsf{u}}}
\def\sv{{\mathsf{v}}}
\def\sA{{\mathsf{A}}}
\def\sB{{\mathsf{B}}}
\def\sC{{\mathsf{C}}}
\def\sF{{\mathsf{F}}}
\def\sP{{\mathsf{P}}}
\def\sQ{{\mathsf{Q}}}
\def\sR{{\mathsf{R}}}
\def\cF{{\mathcal{F}}}
\def\cG{{\mathcal{G}}}
\def\cH{{\mathcal{H}}}
\def\cL{{\mathcal{L}}}
\def\cM{{\mathcal{M}}}
\def\cN{{\mathcal{N}}}
\def\cP{{\mathcal{P}}}
\def\tP{{\mathscr{P}}}



\def\headerproblem #1{\begin{large}\textbf{#1}\end{large}}
\def\func[#1][#2][#3]{#1:#2\rightarrow #3}

\newenvironment{proofsketch}{\noindent\textit{Proof Sketch:}}{\hfill$\square$}

\setlength{\parindent}{15pt}
\frenchspacing
\usepackage{enumitem}
\def\header #1{\noindent\textbf{#1}}

\title{Geometry Personal Notes}
\author{Logan W. Goldberg}
\date{\today}

\begin{document}
\maketitle
\tableofcontents

\chapter{Finite Geometries}
\section{Axiomatic Systems}

This section talks about models and how models imply consistency of a set of axioms. From First-Order Logic, this is simply the Soundness Theorem in action; if there exists a model of $\Sigma$, then $\Sigma$ is consistent. \\

Judith says that an axiomatic system consists of the following components:
\begin{itemize}[nolistsep]
\item Undefined terms
\item Defined terms
\item Axioms
\item A system of logic
\item Theorems\\
\end{itemize}

In this form, Geometry can be considered a many-sorted first order logic system where points, lines, plane, and on are all undefined terms, which are given definitions upon the application of a model. The defined terms are like definable subsets of the universe. In a many-sorted logic system, we can say the set of all parallel lines is defined as witnessed by some or other formula. To avoid circular reasoning, we the axioms are set in place as unchallenged theorems and propositions, that is we accept them without proof. The system of logic is straightforward (e.g., propositional or first order). Theorems are deducible statements given the axioms and a model of said axioms.

Judith says there are two forms of consistency: absolute consistency and relative consistency. Absolute consistency is actually impossible to prove as it requires showing that a system of axioms is self-consistent. G{\"o}del showed in his second Incompleteless Theorem that no system can be shown to be absolutely consistent. Therefore we settle with relative consistency: consistency proved via models. 

When we are provided with a model, the undefined terms are provided with interpretations, not necessarily definitions. This is to say that we can interpret a point for example in a sense up to the axioms. For example, consider the axioms of the Four-Point Geometry Judith gives:\\

\textbf{Four Point Geometry}
\begin{enumerate}[nolistsep]
\item There exist exactly four points.
\item Two distinct points are on exactly one line.
\item Each line is on exactly two points.\\
\end{enumerate}

In this geometry, the undefined terms are 'points', 'lines', and 'on'. Fuck if we know what they are. This is where the models come into play. Suppose we have the model as follows: let $\cP=\{A,B,C,D\}$ be the set of points; there are exactly four points by (1). Let 
\[\cL=\{\{A,B\},\{B,C\},\{C,D\},\{A,C\},\{B,D\},\{A,D\}\}\] 
be the set of lines; notice that two distinct points are on exactly one line (we don't have for example $\{A,B,C\}$ and $\{A,B\}$ as lines), and each line lie on exactly two points. In this instance, we interpret 'on' as contained or contains or basically set inclusion.

Judith then goes on to talk about independence, completeness, and isomorphisms between models; and that basically finishes the section. Let's do three exercises.

\begin{enumerate}
\item Axioms of a Three-Point Geometry
\begin{enumerate}[nolistsep]
\item There exist exactly three points.
\item Two distinct points are on exactly one line.
\item Not all points are on the same line.
\item Two distinct lines are on at least one common point.
\end{enumerate}
Prove that the system is consistent.

\item Is the system independent?
\item Is the system complete?
\item Show that there are exactly six lines in the four point geometry.
\item Prove that any two models of the four-point geometry are isomorphic. 
\end{enumerate}

\textbf{(1)} By (a), let $\cP=\{A,B,C\}$ be the set of points. Let $\cL=\{\{A,B\},\{B,C\},\{A,C\}\}$ be the set of lines. We then interpret 'on' as contains or is contained in. Notice that there are exactly three points in this model and not all points lie on the same line, that is $\{A,B,C\}$ is not a line in this structure. Further, if we take any two points, we can find a unique line they line on, that is if $x,y\in\cP$, then $\{x,y\}$ is the unique line containing the two points. Lastly, if $l,m\in\cL$, then $l$ and $m$ share one point in common since if they did not share a point in common and had two points each, there would be four distinct points, contradicting (a). Hence we have a model of the Three-Point Geometry and thus we've demonstrated relative consistency.

\textbf{(2)} The system is independent. To see this suppose we negate (a). Then a model with four points suffices as follows: let $\cP=\{A,B,C,D\}$ and let $\cL=\{\{A,B,C\},\{A,D\},\{B,D\},\{C,D\}\}$. Notice that every pair of distinct points is on exactly one line, not all points lie on the same line, and any two distinct lines share one point in common.

Now let's negate (b). Let $\cP=\{A,B,C\}$ and let $\cL=\{\{A,B\}\}$ be the set of lines. Notice that we have three points in our structure, not all pairs of distinct points are contained on exactly one line, not all three points are on the same line, and there is only one line in this geometry.

Next let's negate (c). Let $\cP=\{A,B,C\}$ and let $\cL=\{\{A,B,C\}\}$ be the set of lines. Notice that we have three points in our structure, any two distinct points line on exactly one line, there exists a line containing all three points, and there is only one line in this geometry.

Last, let's negate (d). Let $\cP=\{A,B,C\}$ and let $\cL=\{\{A,B\},\{A,C\},\{B,C\},\{C\}\}$. Notice that there are three points in our geometry, each pair of two distinct points is contained on exactly one line, there is no line containing all three points, but there are two lines that do not share a point in common, namely $\{A,B\}$ and $\{C\}$.

Hence our system is independent.

\textbf{(3)} Let $\cM$ and $\cN$ be models of our system. Since they are both models, they both have three points by (1). By (2), both models must have each pair of 2 point subsets of the three points. By (3), both models lack a line containing all three points. By (4), we can remove all singleton lines and the empty line. Therefore both $\cM$ and $\cN$ have just two point lines. From (1.2.3), there are exactly three lines in this geometry. Therefore, 

\textbf{(4)} By (1), there are exactly four points in the four point geometry, so let $\{A,B,C,D\}$ be the points of a geometry. By (2), each subset of two points must be contained in a line. Further by (3), each subset that does not have two points is not a line. Therefore all the lines in this geometry are two element subsets. Now since we are choosing all the two element subsets of a four element, set we apply a little combinatorial fact to find that $\binom{4}{2}=6$. Hence there are six lines in the four-point geometry.

\textbf{(5)}

\section{Finite Projective Planes}
This section discusses the finite projective geometries. A striking difference between the Euclidean plane and the projective plane is that all lines intersect, that is there are no parallel lines. If we tie this back to our incidence geometry axioms, then the projective plane satisfies Playfair's Axiom, but the projective plane fails to be an affine plane.

Judith introduces the axioms of a finite projective plane as follows. The undefined terms are once again, 'point', 'line', and 'incidence'; these will be given meaning when we create a structure which satisfies the axioms. Judith introduces two defined terms as well.\\

\header{Definition.} Points incidence with the same line are said to be \textit{collinear}.\\

\header{Definition.} Lines incidence with the same point are said to be \textit{concurrent}.\\

Below are the projective plane axioms.\\

\begin{enumerate}[label=\textbf{Axiom P\arabic*.}, nolistsep]
\item There exist at least four distinct points, no three of which are collinear.
\item There exist at least one line with exactly $n+1$ distinct points incidence with it.
\item Given two distinct points, there is exactly one line incident with both of them.
\item Given two distinct lines there is at least one point incidence with both of them. \\
\end{enumerate}

Any model satisfying these axioms is called a projective plane of order $n$. On the face of it, these axioms are very daunting. However, I think these axioms are very similar to those of the projective plane axioms Hartshorne gives us in his book. When establishing a model of these axioms, the logic is as follows. By (P1), start with four points. (P3) says to connect each of these points with a line. By (P2), we must have a line with at least three points, but since (P1) says given four points, no three can be collinear, we must add a new point. Now by (P4), each line pair of lines must be incidence at a point. With only five points, we either have that there are no four points, three of which are collinear, or there exists two lines that are parallel. Therefore, add a sixth point. Still, six points is not enough however because there will exist a pair of lines that are parallel. Therefore add a seventh point and connect all the points with three point lines. This construction works and is called the projective plane of order 2 since each line has 3 points and each point lies on 3 lines.

From here, Judith introduces the principle of duality. Given an axiom which describes points and lines, we say the dual of that axiom is simply replacing each instance of point with line and vice versa. For example, the duals of the three point geometry axioms are as follows:\\

\begin{enumerate}[label=\textbf{Axiom \arabic*.}, nolistsep]
\item There exist exactly three lines.
\item Two distinct lines lie on exactly one point.
\item Not all lines are on the same point.
\item Two distinct points are on at least one line.\\
\end{enumerate}

\header{Definition.} An axiomatic system in which the dual of any theorem is also a theorem is said to satisfy the \textit{principle of duality}.

Thus in any axiomatic system, the proof of any theorem can be transformed into a proof of the dual of a theorem by simply dualizing the original proof. To demonstrate that an axiomatic system has the principle of duality, one must show that each axiom's dual is a theorem. It turns out that the finite projective plane axioms satisfy the principle of duality as I shall demonstrate.\\

\header{Theorem.} There exist at least four distinct lines, no three of which are concurrent.
\begin{proof}
Let $\cG$ be a projective geometry of order $n$. By (P2), fix a line $l$ with $n+1$ distinct points; label each of these points $P_1$ through $P_{n+1}$. Since $n>1$, we know that there are at least $n>1$ points on $l$. By (P1), fix a point $Q_1$ and $Q_2$ not on $l$, and then by (P3), let $m_1$ and $m_2$ be two pairs of distinct lines for which $m_1$ is incidence with $P_1$ and $Q_1$ and $P_2$ and $Q_2$. Now by (P3), let $m_3$ be the line containing $Q_1$ and $Q_2$. Notice that $l$ is parallel to $m_3$ and $m_1$ is parallel to $m_2$. Using combinatorial reasoning, we are choosing 3 lines from four possible lines, resulting in $\binom{4}{3}=4$. Here are all the possible configurations of lines that can be incident at a point:
\[\{l,m_1,m_2\}, \{l,m_1,m_3\}, \{l,m_2,m_3\}, \{m_1,m_2,m_3\}.\]
Notice in general no matter how we pick these three lines, it is impossible to choose three such that they are concurrent because we must choose two that are parallel. Thus it is the case that that we have four lines, no three of which are concurrent. 
\end{proof}

\header{Theorem.} There exists at least one point with exactly $n+1$ distinct lines incident with it.

\begin{proof}
Suppose that $\cG$ is a model of a finite projective geometry of order $n$. By (P2), fix a line $m$ such that $m$ has $n+1$ points; label each point $P_1,P_2,\ldots,P_n,P_{n+1}$. By (P1), fix a point $P$ not on $l$. Now (P3) allows us to create, lines $l_1,l_2,\ldots,l_{n+1}$ such that each $l_i$ is incident with $P_i$ and $P$. Now we must argue that each of these lines is distinct. Suppose that for some $1\leq i\neq j\leq n+1$ that $l_i=l_j$. Then $P_i$ and $P_j$ are incident with $l_i$ and $l_j$. By (P3), it follows that $l=l_i=l_j$ since there for any two pair of distinct points, we have a unique line containing them both. As such, $P$ is incident with $l$ since $P$ is incident with $l_i$, a contradiction. Therefore $l_i\neq l_j$ and thus we have $n+1$ distinct points. 

Now we must demonstrate that there are no more than $n+1$ distinct lines incident with $P$. Suppose that there is a line $l_{n+2}$ incident with $P$. Then by (P4), $l_{n+2}$ must intersect $l$ at a point, call it $Q$. Since $l$ has $n+1$ many points, $Q$ must equal one of $P_1,P_2,\ldots,P_{n+1}$. Suppose without loss of generality that $Q=P_1$. Then by (P3), since $P_1$ and $P$ are incident with $l_1$ and $l_{n+2}$, it follows that $l_1=l_{n+2}$. Hence there exists at least one point with exactly $n+1$ distinct lines incident with it.
\end{proof}

\header{Theorem.} Given two distinct lines, there is exactly one point incident with both of them.

\begin{proof}
Let $l$ and $m$ be two distinct lines. From (P4), we know there is at least one point $P$ incident with both $l$ and $m$. Suppose there are two points incident with both $l$ and $m$, call it $Q$. If $Q$ is distinct from $P$, then by (P3), it follows that $l=m$, contradicting our assumption that $l$ and $m$ were distinct. Therefore $P=Q$ and thus for each pair of distinct lines, there is exactly one point incident with both.
\end{proof}

\header{Theorem.} Given two distinct points there is at least one line incidence with both of them.

\begin{proof}
Follows immediately from (P3).
\end{proof}

Therefore it is the case that the axioms of a finite projective geometry satisfy the principle of duality. Let's show this property in action.

\header{Theorem.} In a projective plane of order $n$, each point is incident with exactly $n+1$ lines.
\begin{proof}
Let $P$ be a point in the plane. By (P2), fix a line $l$ with points $P_1,P_2,\ldots P_{n+1}$ incident with $l$. We now have two cases to consider: when $P$ is incident with $l$ and when $P$ is not incident with $l$. 

Suppose first that $P$ is not incident with $l$. Since $P$ is not incident with $l$, (P3) tells us that any pair of distinct points, namely $P_i$ with $P$ has a distinct line containing both $P_i$ and $P$. Therefore $P$ has at least $n+1$ lines. By the proof of the dual of (P2), it follows that $P$ has exactly $n+1$ lines incident with it.

Suppose that $P$ is incident with $l$. Then since $l$ has $n+1$ distinct points, $P$ must be one of those $n+1$ points; suppose without loss of generality $P=P_1$. By (P1), it follows that there exists points $Q$ not on $l$. By the dual of (P1), let $m$ be a line which is not incident with either $P$ nor $Q$. By (P4), let $P_{n+1}$ be the point incident to $m$. Since $Q$ is not on $m$ nor $l$, $Q$ is incident with $n+1$ lines by the previous case; let each line be $m_1,m_2,\ldots,m_{n+1}$. Since $m$ and $m_i$ are distinct for each $i$, we have by (P4) that $m$ and $m_i$ are incident at a point $R_i$ for each $i$. If $R_i=R_j$ for some $i\neq j$, then since $m_i$ and $m_j$ are distinct, we have that $m_i$ and $m_j$ are incident at two points, contradicting the dual of (P3); hence $m$ has at least $n+1$ points. Suppose now that $m$ has a point $R_{n+2}$. By (P3), it follows that there exists $m_2$ which is incident with both $R_{n+2}$ and $Q$. Since $Q$ is incident with exactly $n+1$ lines, we can conclude that $m_{n+2}=m_i$ for some $1\leq i\leq n+1$. Since $P=P_1$, $P_1$ is incident with $l$, and $l$ and $m$ intersect at $P_{n+1}$, we conclude that $P$ is not on $l$. Therefore by case 1, since $P$ is not incident with $m$, we conclude that $P$ is incident with exactly $n+1$ lines.
\end{proof}

With this theorem behind us, we may use the principle of duality to conclude the following theorem:\\

\header{Theorem.} In a projective plane of order $n$, each line is incident with exactly $n+1$ points.\\

Another application of the principle of duality can be seen in the following combinatorial argument:\\

\header{Theorem.} In a project plane of order $n$, there are exactly $n^2+n+1$ points and $n^2+n+1$ lines.
\begin{proof}
Let $P$ be a point in the projective plane of order $n$. By a previous theorem, $P$ is incident with exactly $n+1$ lines. By the above theorem, we have that each of these lines is incident with $n+1$ points, including $P$. Since each of these lines contains $P$, we have $n+1$ lines with $n$ distinct points each. We then have that $(n+1)n+1=n^2+n+1$ many points. The dual argument verifies that there are $n^2+n+1$ lines.
\end{proof}

Judith concludes the section by discussing the existence of orders of projective planes. We know currently that every projective plane of order $p^\alpha$ where $p$ is prime exists. However, it remains an open question as to the other non-prime power orders.

\begin{enumerate}
\item[] \textbf{Axioms for Finite Affine Planes}
\begin{enumerate}[label=\textbf{Axiom A\arabic*.}]
	\item There exist at least four distinct points, no three of which are collinear.
	\item There exists at least one line with exactly $n>1$ points on it.
	\item Given two distinct points, there is exactly one line incident with both of them.
	\item Given a line $l$ and a point $P$ not on $l$, there is exactly one line through $P$ that does not intersect $l$.
\end{enumerate}
\item Show that a finite affine plane does not satisfy the principle of duality.
\item In an affine plane of order $n$, each point lies on exactly $n+1$ lines.
\item In an affine plane of order $n$, there are exactly $n^2$ points and $n^2+n$ lines.
\end{enumerate}

\section{Desargues' Configurations}
Visiting one more type of finite geometry, we have what's known as The Desargues' Configurations. These configurations satisfy the principle of duality as well as mimic a relation in projective geometry known as \textit{polarity}. We call these structures configurations as opposed to geometries because there may be points that are not on lines. This all culminates in a theorem known as Desargues Theorem which talks about two sets of noncollinear points being perspective to a point or a line. If two triangles $\Delta ABC$ and $\Delta DEF$ that lines joining corresponding sides are congruent, then the triangles are said to be perspective from a point. The dual argument says that if the intersection of corresponding sides are collinear, then the two triangles are perspective from a line.\\

\header{Theorem.} (Desargues' Theorem) \textit{If two triangles are perspective from a point, then they are perspective from a line.}\\

\noindent Judith does not provide a proof of this theorem, but we will revisit it later in chapter 4.\\

\noindent\begin{large}\begin{center}
\textbf{Axioms of Desargues' Configurations}
\end{center}\end{large}

\noindent\textbf{Undefined Terms:} Point, line, on

\noindent\textbf{Defined Terms:} If there are no lines joining a point $M$ to a line $m$, then $m$ is called a \textit{polar} of $M$ and $M$ is called a pole of $m$.

\begin{enumerate}[nolistsep, label=\textbf{Axiom D\arabic*.}]
\item There exists at least one point.
\item Each point has at least one polar.
\item Every line has at most one pole.
\item Two distinct points are on at most one line.
\item There are exactly three distinct points on every line.
\item If line $m$ does not contain a point $P$, then there is a point on $m$ and any polar of $P$.
\end{enumerate}




\chapter{Non-Euclidean Geometries}
\section{Euclid's Geometry}
\section{Non-Euclidean Geometry}
\section{Hyperbolic Geometry -- Sense Parallels}
\section{Hyperbolic Geometry -- Asymptotic Triangles}
\section{Hyperbolic Geometry -- Saccheri Quadrilaterals}
\section{Hyperbolic Geometry -- Area of Triangles}
\section{Hyperbolic Geometry -- Ultraparallels}
\section{Elliptic Geometry}
\section{Significance of Non-Euclidean Geometries}

\chapter{Geometric Transformations of Euclidean Plane}
\section{An Analytic Model of the Euclidean Plane}
\section{Linear Transformations of the Euclidean Plane}
\section{Isometries}
\section{Direct Isometries}
\section{Indirect Isometries}
\section{Symmetry Groups}
\section{Similarity Transformations}
\section{Affine Transformations}

\chapter{Projective Geometry}
\section{The Axiomatic System and Duality}
\section{Perspective Triangles}
\section{Harmonic Sets}
\section{Perspectives and Projectivities}
\section{Conics In the Projective Plane}
\section{An Analytic Model for the Projective Plane}
\section{The Analytic Form of Projectivities}
\section{Cross Ratios}
\section{Collineations}
\section{Correlations and Polarities}
\section{Subgeometries of Projectice Geometry}


\end{document}